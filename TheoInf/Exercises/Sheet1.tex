\documentclass[8pt]{extreport}
\usepackage{parskip}
\usepackage{amsmath}
\usepackage{amssymb}
\usepackage{fancyhdr}
\usepackage{enumitem}
\usepackage{geometry}
\geometry{a4paper, margin = 1in}
\setlength{\headheight}{15.2pt}
\pagestyle{fancy}
\fancyhead[L]{Theoretische Informatik Sheet 1}
\fancyhead[R]{Nicolae Binica, Tobias Böschung, Gregory Rozanski}
\title{Theoretische Informatik Sheet 1}
\begin{document}
\begin{enumerate}
\item
\begin{enumerate}[label = \alph*)]
\item We start by defining the base cases i.e $n\leq 2$. These are all 0 because the size of a word which contains each of the three letters at least once must be greater than 2. For $n \geq 3$ we can use the principle of inclusion and exclusion. $3^n$ is the total number of words which can be built using the three letters, $3\cdot2^n$ are the words which are made only using 2 of the three letters. Finally we add the 3 words which formed only using one letter to compensate for doubles that were subtracted in the second term.
\begin{itemize}
\item n = 0 $\Rightarrow$ 0
\item n = 1 $\Rightarrow$ 0
\item n = 2 $\Rightarrow$ 0
\item n $\geq 3 \Rightarrow 3^n-3\cdot 2^n + 3$ 
\end{itemize}
\item We define:
\begin{itemize}
\item $a_n:= \#$ of words which end in 1 with length n and do not contain 11 as a subword.
\item $b_n:= \#$ of words which end in 0 with length n and do not contain 11 as a subword
\end{itemize}
The solution to are task is: $sol_n:= a_n + b_n$
We now derive the following two formulas:
\begin{enumerate}
\item $b_n = b_{n-1} + a_{n-1} = sol_{n-1}$\\ because we can append a 0 or a 1 to a word of size n-1 which ends in 0.
\item $a_n =b_{n-1} = b_{n-2}+a_{n-2} = sol_{n-2}$\\  because words of size n-1 must end in a 0 for us to be able to append a 1

\end{enumerate}
for $n\geq 3$\\
$\Rightarrow$
\begin{itemize}
\item $sol_{0}  = 0$
\item $sol_{1} = 2$
\item $sol_{2} = 3$
\item $sol_{n \geq  3} = sol_{n-2} + sol_{n-1}$
\end{itemize}
\end{enumerate}
\item 
\begin{enumerate}[label = \alph*)]
\item
We assume there is a non-empty finite language $L \neq \{\lambda\}$ satisfying $L^2 = L$ and hence it must contain a largest element x according to the canonical ordering.\\
$\Rightarrow$ by definition of concatination there is an element $x' \in L^2$ such that $ x' = x\cdot x$ (\\
$\Rightarrow$ since $|x'| > |x|$ and x was the largest element in L it follows that $x' \notin L$\\
$\Rightarrow$ No non-empty finite Language L exists such that $L \neq \{\lambda\}$ and $L^2 = L$
\item We define the three languages as follows:
\begin{itemize}
\item $L_1 = \{0\}^*$
\item $L_2 = \{0\}$
\item $L_3 = \{00\}$
\end{itemize}
$\Rightarrow  (L_2 \cap L_3) = \{0\} \cap \{00\} = \emptyset$\\
$\Rightarrow L_1 \cdot \emptyset = \emptyset$\\
$\Rightarrow L_1 \cdot (L_2 \cap L_3)$ is finite\\
\newline
$L_1L_2 = L_1^+$ and $L_1L_3 = L_1\backslash\{\lambda,0\}$\\
$\Rightarrow L_1^+ \cap L_1\backslash\{\lambda,0\} = L_1\backslash\{\lambda,0\}$ \\
$\Rightarrow L_1\backslash\{\lambda,0\}$ is by definition infinite
\end{enumerate}
\item We must prove: An infinite language L is recursive $\iff$ there is an algorithm enumerating L\\
$"\Rightarrow"$\\
Assume L is recursive, then there exists and Algorithm A which solves the Decision Problem (Entscheidungsproblem)\\
$\Rightarrow$ We iterate through each element $x \in \sum^*$ in canonical order and decide with A if we add it to our enumeration.\\
$\Rightarrow$ There is an algorithm enumerating L
\newline
$"\Leftarrow"$\\
Assume there is an algorithm A which enumerates L\\
We define an algorithm B which, when given an arbitrary $x \in \sum^*$ goes through the enumeration in canonical order and checks if x is equal to the current element e in the enumeration. If $|x| > |e|$ then B outputs "x not in L". Since $|x| < \infty$, B will terminate in a finite amount of time. If x == e then we output "x in L".\\
$\Rightarrow$ B solves the Decision Problem for L\\
$\Rightarrow$ L is recursive
\end{enumerate}
\end{document}