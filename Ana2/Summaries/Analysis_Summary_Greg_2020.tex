\documentclass[8pt]{extreport}
\usepackage{parskip}
\usepackage{amsmath}
\usepackage{amssymb}
\usepackage{graphicx}
\usepackage{subcaption}
\usepackage{enumitem}
\usepackage{geometry}
\usepackage{float}
\geometry{a4paper, margin=1in}
\title{Analysis II\\Summary}
\begin{document}
	\maketitle
	\newpage

\chapter{Ordinary differential equations}
\section{Differential Equation:} 
An equation for a function f that relates the values of f at x, f(x) to the values of its derivatives at the same point x. We distinguish between the number of variables present in the function:
\begin{itemize}
\item \underline{\textbf{One variable:}} Ordinary differential equations (ODE)
\item \underline{\textbf{Several Variables:}} Partial differential equations (PDE)
\end{itemize}

\underline{Examples:}
\begin{itemize}
\item $f'(x) = f(x)$
\item $f''(x) = -f(x)$
\end{itemize}

\underline{Notation:} We write $y,y',y'',y^{(3)},...$ instead of $f(x),f'(x),f''(x),f^{(3)}(x)$ 

\textbf{\underline{Order:}} The largest derivative present in the equation. Examples:
\begin{itemize}
\item $y' = 2xy$ order 1
\item $y^{(3)} + 2xy'' + e^xy +1 = 0$ order 3
\end{itemize}


The solution to an ODE is not unique in general. When given initial conditions then we can find unique solutions. E.g:
\begin{center}
$y' = x+1$\\
$y = \frac{x^2}{2} + x + c$
\end{center}
is a solution for any c. If we are also given y(0) = 1 then c= 1 is a unique solution.

\section{Linear Differential equations}

A linear ODE of order k on an interval $I \subset \mathbb{R}$ is an eqn of the form:
\begin{center}
$y^{(k)} + a_{k-1}(x)y^{(k-1)} + ... + a_1(x)y' + a_0(x)y = b(x)$
\end{center}
where a(x) and b(x) are continuous functions from $I \text{ to } \mathbb{C}$.\\
For a linear ODE the following hold:
\begin{itemize}
\item y and all its derivatives appear in order 1
\item there are no products of the function y  and its derivatives
\item neither the function nor its derivatives are inside another function e.g $\sqrt{y}$, sin(y),...
\end{itemize}
If b = 0 then we say the equation is \textbf{homogeneous} otherwise \textbf{inhomogeneous}\\


Solving a linear ODE means finding all functions $f:I \rightarrow \mathbb{C}$ that are k times differentiable such that $\forall x \in  I$ the function satisfies the differentiable equation.

\textbf{\underline{Initial Condition}} A set of equations specifying the values of the derivatives at some initial point. 

\underline{\textbf{Theorem 2.2.3}} Let $I\subset \mathbb{R}$ and open interval $k\geq 1$ and integer. Consider the linear ODE
\begin{center}
$y^{(k)} + a_{k-1}(x)y^{(k-1)} + ... + a_1(x)y' + a_0(x)y = b(x)$
\end{center}
where coefs $a_i(x), b(x)$ are continous functions
\begin{enumerate}
\item Let $S_0$ be the set of solutions for b=0, then $S_0$ is a vector space of dimension k.
\item For any initial conditions, i.e for any choice of $x_0 \in I$ and $(y_0,...,y_{k-1}) \in \mathbb{C}^k$ there is a unique solution $f \in S$ such that $f_(x_0) = y_0, ... f^{(k)}(x_0)= y_k$
\item For an arbitrary b the set of solutions of the linear ODE is $S_b = \{f + f_p | f \in S_0\}$ where $f_p$ is one \textbf{particular} solution. $S_b$ is not a vector space. 
\item For any initial condition there is a unique solution.
  
\end{enumerate}

The linearity of the diff equation alsos implies a \textbf{superposition} principle. Suppose we have 2 different functions $b_1(x), b_2(x)$ on the RHS with solutions $f_1,f_2: Df_1 = b_1, Df_2 =b_2$ then $f_1 + f_2$ solves $Df = b_1+b_2$ \\

Given a diff eqn and a possible solution we can always verify whether it is indeed a solution or not. 

\section{Linear differential equations of order 1}

We consider y'+ay = b, where a,b are continous functions. 2 steps:
\begin{itemize}
\item Find solutions of the corresponding homogeneous equation y' + ay = 0.
\item Find a particular solution $f_p:I \rightarrow \mathbb{C}$ such that $f_p + a f_p = b$ 
\end{itemize}

If f is a solution then so is zf for any constant $z \in \mathbb{C}$

\underline{Homogeneous solution:}
$y' + ay = 0$\\
$\Rightarrow y' = -ay$\\
$\Rightarrow \frac{y'}{y} = a$\\
$\Rightarrow \int \frac{y'(x)}{y(x)}dx = -\int a(x) dx := A(x)$\\
$\Rightarrow ln|y(x)| = -A(x) +c$\\
$\Rightarrow y = z\cdot e^{-A(x)}$ for some constant z\\

\underline{Solution of inhomogeneous equation}
$y' + ay = b$\\
There are two methods to solve this:
\begin{itemize}
\item Educated guess: the LHS tries to imitate the RHS i.e if b(x) is a polynomial we guess that $f_p$ is also a polynomial or if b is a trig function then we guess $f_p$ is also a trig function
\item Variation of constants: Assume
\begin{center}
$ f_p = z(x)e^{-A(x)}$ 
\end{center}
for some function $z:I \rightarrow \mathbb{C}$. We then put this into the equation and see what it forces z(x) to satisfy
\end{itemize} 


The same particular solution can also be obtained by the method of \textbf{Integration factor} (IF). Given a ODE of the following form:
\begin{center}
$\frac{dy}{dx} + a(x)y = b(x)$
\end{center}
one multiplies both sides of the equation by an IF of:
\begin{center}
$e^{\int a(x)dx}$
\end{center}
$\Rightarrow$
\begin{center}
$\frac{dy}{dx}e^{\int a(x)dx} + a(x)ye^{a(x)dx} = b(x)e'{\int a(x) dx}$
\end{center}
The left hand side simplifies to:
\begin{center}
$\frac{d}{dx}(ye^{\int a(x)dx}) := z(x)$\\
$\Rightarrow y = z(x)e^{-A(x)}$
\end{center}
$\Rightarrow$
\begin{center}
$z'(x) = b(x)e^{\int a(x) dx} = b(x) e^{A(x)}$
\end{center}
\underline {Example:}
\begin{center}
$ x \frac{dy}{dx} -2y = x^2$
\end{center}
Assume $x \neq 0$. We now put the equation in the above form.
\begin{center}
$\frac{dy}{dx} - \frac{2}{x}y = x$
\end{center}
\begin{itemize}
\item $a(x) = \frac{-2}{x}$
\item $b(x) = x$
\item $A(x) = -2\int\frac{1}{x}dx = -2ln(x) = ln(x)^{-2}$
\item $e^{A(x)} = e^{lnx^{-2}} = \frac{1}{x^2}$
\end{itemize}
$z'(x) = b(x) e^{A(x)} = x \cdot \frac{1}{x^2} = \frac{1}{x}$\\
$\Rightarrow z(x) = ln(x)$\\
\begin{itemize}
\item $y_h = ze^{-A(x)} = zx^2$
\item $y_p =z(x)e^{-A(x)} = ln(x)x^2$
\end{itemize}
$\Rightarrow y = y_p +y_h = x^2ln(x) + zx^2$

\section{Linear differential equations with constant coefficients}

 For a linear ODE with constant coefficients
\begin{center}
$y^{(k)} + a_{k-1}y^{k-1} + ... + a_0y = 0$
\end{center}
The Polynomial
\begin{center}
$P(\lambda) = \lambda^{k} + a_{k-1}\lambda^{k-1} + ... + a_0$
\end{center}
is called the \textbf{companion/charateristic polynomial} of the equation. The zeroes of $P(\lambda)$ are called the \textbf{eigenvalues}\\
\underline{Example:}\\
$y''-y = 0$\\
$\Rightarrow P(\lambda) = \lambda^2-1 = (\lambda- 1)(\lambda + 1)$\\
$\Rightarrow$ 2 solutions: $e^{-x},e^{x}$\\
Any solution of the equation are of the form
\begin{center}
$y(x) = z_1e^{-x} + z_2e^{x}$
\end{center}

\underline{\textbf{Theorem}} let $\lambda_1,...\lambda_r$ be pairwise distince eigenvalues of $P(\lambda)$, characteristic polynomial of 
\begin{center}
$(*)\quad y^k + a_{k-1}y^{k-1} + ... + a_0y = 0$
\end{center}
with corresponding multiplicities $m_1,...,m_r$ Then the functions
\begin{center}
$f_{j,l} :\mathbb{R} \rightarrow \mathbb{C} \quad x \mapsto x^le^{\lambda_jx}$
\end{center}
for $1 \leq j \leq r, 0 \leq l < m_j$\\
form a system of solutions of the homogeneous D.E (*).\\
\newline
\underline{Example}
$y'' -2y' + 1 = 0$\\
$\Rightarrow P(\lambda) = \lambda^2 - 2 \lambda + 1 = (\lambda -1)^2$\\
$\Rightarrow \lambda = 1$ has multiplicity of 2\\
$\Rightarrow$ the solutions are $e^x, xe^x$















\end{document}