\documentclass[8pt]{extreport}
\usepackage{parskip}
\usepackage{amsmath}
\usepackage{amssymb}
\usepackage{graphicx}
\usepackage{subcaption}
\usepackage{enumitem}
\usepackage{geometry}
\usepackage{float}
\geometry{a4paper, margin=1in}
\title{Visual Computing\\ Summary}

\begin{document}
	\maketitle
	\newpage
\chapter{The Digital Image}

\section{What is an image:}

\paragraph{\underline{Signal:}} A function depending on some variable with physical meaning

\paragraph{\underline{Image:}} A continuous function. The value can take on physical values. e.g Brightness, temperature, pressure, depth, etc. We distinguish images with:
\begin{itemize}
\item \textbf{2 variables:} xy- coordinates
\item \textbf{3 variables:} xy + time (video)
\end{itemize}
Hence an image is a picture or pattern of a value varying in space and/or time.
\begin{center}
$f: \mathbb{R}^n \rightarrow S$
\end{center}

\section{The digital camera}
\begin{figure}[H]
\centering
\includegraphics[width = 70mm]{VC1.png}
\end{figure}


\paragraph{\underline{The sensor array:}} An array of photosites. Each photosite is a bucket of electrical charge and contain a charge proportional to the incident light intensity during exposure.

\paragraph{\underline{Analog to Digital Conversion:}} The ADC measures the charge and digitizes the result. The conversion happens line by line. The charges in each photosite move down through the sensor array.

\begin{figure}[H]
\centering
\includegraphics[width = 70mm]{VC2.png}
\end{figure}

Various errors:
\begin{itemize}
\item \textbf{Blooming:} The buckets have finite capacity, saturation causes blooming. It happens when a large amount of light gets focused to a single point on your cameras image sensor. This can create so much charge that it actually bleeds from pixel to pixel until it eventually spreads out.

\begin{figure}[H]
\centering
\includegraphics[width = 40mm]{VC3.png}
\end{figure}
\item \textbf{Bleeding or Smearing:} During transit buckets still accumulate some charges. The amount is influenced by the time "in transit" versus the integration time
\begin{figure}[H]
\centering
\includegraphics[width = 40mm]{VC5.png}
\end{figure}
\item \textbf{Dark Current:} A relatively small electric current that flows through photosensitive devices even when no photons are entering the device.
\begin{figure}[H]
\centering
\includegraphics[width = 40mm]{VC4.png}
\end{figure}
\end{itemize}

\paragraph{\underline{CMOS:}} Contains the same sensor elements as CCD but each photo sensor has its own amplifier. the benefits:
\begin{itemize}
\item Recent technology
\item Standard IC technology
\item Cheap
\item Low Power
\item Less sensitive
\item Per pixel amplification
\item Random pixel access
\item Smart pixels
\item On chip integration with other components
\end{itemize}

\section{Sampling 1D:}

Sampling in 1D takes a function, and returns a vector whose elements are values of that function at the sample points.

\begin{figure}[H]
\centering
\begin{subfigure}[b]{0.3\linewidth}
\includegraphics[width = \linewidth,scale = 1]{VC6.png}
\end{subfigure}
\begin{subfigure}[b]{0.3\linewidth}
\includegraphics[width = \linewidth,scale = 1]{VC7.png}
\end{subfigure}
\end{figure}
\subsection{Reconstruction:} Making samples back into a continuous function. This amounts to guessing what the function did in between the individual samples

\subsection{Undersampling:} Occurs if not enough sampling points are available. It results in a loss of information and can be indistinguishable from other samples e.g undersampling a sin wave:
\begin{figure}[H]
\centering
\begin{subfigure}[b]{0.3\linewidth}
\includegraphics[width = \linewidth,scale = 1]{VC8.png}
\end{subfigure}
\begin{subfigure}[b]{0.3\linewidth}
\includegraphics[width = \linewidth,scale = 1]{VC9.png}
\end{subfigure}
\end{figure}


\section{Sampling 2D:}

Sampling in 2D takes a function and returns an array. The array can have infinite dimensions and have negative as well as positive indices.

\subsection{Reconstruction continuous signals:}
\begin{figure}[H]
\centering
\begin{subfigure}[b]{0.3\linewidth}
\includegraphics[width = \linewidth,scale = 1]{VC10.png}
\end{subfigure}

\end{figure}

\subsection{ Nyquist Frequency:} Half the sampling frequency of a discrete signal processing system. The signals max frequency (bandwidth) must be smaller than this.

\subsection{ Quantization:} Real valued function gets digital values (integer values). Quantization is lossy i.e the orignial signal cannot be reconstructed anymore. Simple quantization uses equally spaced levels with k intervals:
\begin{center}
$ k= 2^b$
\end{center}
\begin{figure}[H]
\centering
\begin{subfigure}[b]{0.3\linewidth}
\includegraphics[width = \linewidth,scale = 1]{VC11.png}
\end{subfigure}
\begin{subfigure}[b]{0.3\linewidth}
\includegraphics[width = \linewidth,scale = 1]{VC12.png}
\end{subfigure}
\begin{subfigure}[b]{0.3\linewidth}
\includegraphics[width = \linewidth,scale = 1]{VC13.png}
\end{subfigure}
\end{figure}

\subsection{Image Properties:}

\paragraph{\underline{Geometric Resolution:}} How many pixels per area

\paragraph{\underline{Radiometric Resolution:}} How many bits ber pixel

\section{Image Noise}

The different types of common Noise models:

\begin{itemize}
\item \textbf{\underline{Gaussian noise:}}
\begin{center}
$I(x,y) = f(x,y) + c$
\end{center}
 where $c \sim N(0,\sigma^2)$ so that 
\begin{center}
$p(c) = (2\pi\sigma^2)^{-1}e^{\frac{-c^2}{2\sigma^2}}$
\end{center}

\item  \textbf{\underline{Poisson noise:}}
\begin{center}
$p(k) = \frac{\lambda^k e^{-\lambda}}{k!}$
\end{center}
\item \textbf{\underline{Rician noise:}}
\begin{center}
$p(I) = \frac{I}{\sigma^2}e^{\frac{-(I^2 + f^2)}{2\sigma^2}}I_0(\frac{If}{\sigma^2}$
\end{center}
\item \textbf{\underline{Multiplicative noise:}}
\begin{center}
$I = f + fc$
\end{center}
\end{itemize}

\paragraph{\underline{Signal to noise ration (SNR)}} An index of image quality:
\begin{center}
$s = \frac{F}{\sigma}$ where $F= \frac{1}{XY}\displaystyle\sum_{x=1}^{X}\displaystyle\sum_{y=1}{Y}f(x,y)$
\end{center}

\paragraph{\underline{Peak Signal to Noise Ration (PSNR)}}
\begin{center}
$s_{peak} = \frac{F_{max}}{\sigma}$
\end{center}

\section{Color Cameras:}

\paragraph{\underline{Prism Color Camera:}} Seperates the light into 3 beams using a dichroic prism
\begin{figure}[H]
\centering
\begin{subfigure}[b]{0.3\linewidth}
\includegraphics[width = \linewidth,scale = 1]{VC14.png}
\end{subfigure}
\end{figure}
\paragraph{\underline{Filter mosaic:}} Filter is coated directly on sensor

\paragraph{\underline{Filter wheel:}} Rotate multiple filters in front of lens. This is only suitable for static scenes.
\begin{figure}[H]
\centering
\begin{subfigure}[b]{0.3\linewidth}
\includegraphics[width = \linewidth,scale = 1]{VC15.png}
\end{subfigure}
\end{figure}

\chapter{Image Segmentation:}

Image segmentation is the concept of partitioning an image into regions of interest.

\paragraph{\underline{Complete Segmentation:}} A finite set of regions $R_1, ..., R_N$, such that
\begin{center}
$I = \bigcup\limits_{i=1}^{N}R_i$ and  $R_i \cap R_j = \phi \quad \forall i\neq j$
\end{center}
Where I is an image.

\section{Thresholding}

Thresholding is a simple segmentation process which produces a binary image B. It labels each pixel "in" or "out" of the region of interest by comparison of the greylevel with a threshold T:
\[ 
B(x,y) =
\begin{cases}
1 \quad \text{ if } I(x,y) \geq T\\
0 \quad \text{ if } I(x,y) < T\\
\end{cases}
\]

\begin{figure}[H]
\centering
\begin{subfigure}[b]{0.3\linewidth}
\includegraphics[width = \linewidth,scale = 1]{VC16.png}
\end{subfigure}
\begin{subfigure}[b]{0.3\linewidth}
\includegraphics[width = \linewidth,scale = 1]{VC17.png}
\end{subfigure}
\begin{subfigure}[b]{0.3\linewidth}
\includegraphics[width = \linewidth,scale = 1]{VC18.png}
\end{subfigure}
\begin{subfigure}[b]{0.3\linewidth}
\includegraphics[width = \linewidth,scale = 1]{VC19.png}
\end{subfigure}
\end{figure}

\section{ROC Analysis}

ROC = Recieveer Operating Characteristic\\

An ROC curve characterizes the performance of a binary classifier. A binary classifier distinguishes between two different types of things e.g:
\begin{itemize}
\item Healthy/afflicted patients 
\item Pregnancy tests
\item Object detection
\item Foreground/background image pixels
\end{itemize}

\subsection{Classification Error}
Binary classifiers make errors. There are two inputs to a binary classifier i.e positives and negatives but there are four possible outcomes in any test:
\begin{figure}[H]
\centering
\begin{subfigure}[b]{0.5\linewidth}
\includegraphics[width = \linewidth,scale = 1]{VC20.png}
\end{subfigure}
\end{figure}

\subsection{ROC Curve}
Characterizes the error trade-off in binary classification tasks. It plots the TP (True-positive) against the FP (False-positive) fraction

\begin{center}
TP fraction (sensitivity): $\frac{\text{ True positive count } }{P}$\\
FP fraction (1-specificity):$ \frac{\text{ False positive count } }{N}$
\end{center}

An ROC curve always passes through (0,0) and (1,1)\\
We choose an operating point by assigning a relative costs and values to each outcome:
\begin{itemize}
\item $V_{TN}$ Value of true negative
\item $V_{TP}$ Value of a true positive
\item $C_{FN}$ Cost of a false negative
\item $C_{FP}$ Cost of a false positive
\end{itemize}

We choose the point on the ROC curve with gradient:
\begin{center}
$\beta = \frac{N}{P}\frac{V_{TN} + C_{FP}}{V_{TP} + C_{FN}}$
\end{center}
For simplicity we often set $V_{TN} = V_{TP} = 0$
\begin{figure}[H]
\centering
\begin{subfigure}[b]{0.5\linewidth}
\includegraphics[width = \linewidth,scale = 1]{VC21.png}
\end{subfigure}
\end{figure}


In reality we use 2-3 seperate sets of test data:
\begin{enumerate}
\item \textbf{Training set:} For tuning the algorithm
\item \textbf{Validation set:} For tuning the performance score
\item \textbf{Test set:} To get a final performance score on the tuned algorithm
\end{enumerate}

\section{Pixel Connectivity:}

\paragraph{\underline{Pixel Neighbourhood:}}
\begin{figure}[H]
\centering
\begin{subfigure}[b]{0.5\linewidth}
\includegraphics[width = \linewidth,scale = 1]{VC22.png}
\end{subfigure}
\end{figure}

\paragraph{\underline{Pixel Paths:}}
\begin{itemize}
\item A 4-connected path between pixels $p_1$ and $p_n$ is a set of pixels $\{p_1,p_2,...,p_n\}$ such that $p_i$ is a 4-neighbour of $p_{i+1}$ i=1,...,n-1
\item A  8-connected path, $p_i$ is an 8-neighbour of $p_{i+1}$
\end{itemize}

\paragraph{\underline{Connected Regions:}}
\begin{itemize}
\item A region is 4-connected if it contains a 4-connected path between any two of its pixels
\item A region is 8-connected if it contains an 8-connected path between any two of its pixels
\end{itemize}

\section{Image Labelling}

\paragraph{\underline{Connected components Labelling:}} Labels each connected component of a binary image with a seperate number
\begin{figure}[H]
\centering
\begin{subfigure}[b]{0.5\linewidth}
\includegraphics[width = \linewidth,scale = 1]{VC23.png}
\end{subfigure}
\end{figure}

\paragraph{\underline{Foreground Labelling:}} Only extract the connected components of the foreground
\begin{figure}[H]
\centering
\begin{subfigure}[b]{0.5\linewidth}
\includegraphics[width = \linewidth,scale = 1]{VC24.png}
\end{subfigure}
\end{figure}










 \end{document}